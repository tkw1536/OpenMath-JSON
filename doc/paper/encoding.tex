\section{The OpenMath-JSON encoding}

\subsection{Architecture}

\subsection{Encoding Details}

\subsection{The Demo Site}

To demonstrate our OpenMath-JSON encoding, we have created a demo site which can be found at~\cite{openmathjson:web}. 
This site is implemented in TypeScript\ednote{cite typescript} and encapsulated using Docker\ednote{cite docker}. 

The demo site serves three purposes.

Primarily, it serves as a presentation of the encoding, providing examples and documenting it's usage. 

Secondly, it enables validation of OpenMath JSON objects. This can be seen in Figure\ednote{Make figure}.
The user can enter some JSON, press the \textit{Validate JSON} button, and receive immediate feedback if their JSON is a valid OpenMath object or not. 
In particular, the user can also see a detailed error message if their object is not valid OpenMath JSON. 

This makes use of the OpenMath JSON schema, and validates the users' JSON using a generic JSON Schema Validator. 
Furthermore, this is also exposed using a REST API, enabling easy validation of OpenMath JSON in other applications. 

Thirdly, it enables translation between XML and JSON OpenMath objects. 
Like for validation, the site enables the user to enter some JSON and be presented with some XML and vice-versa. 
This can be seen in Figure\ednote{Make figure + screenshot}.

As we designed our encoding with this translatability goal in mind, the implementation of it was straight-forward. 
Nonetheless, this translation is also exposed using a REST API. 