\section{The OpenMath-JSON encoding}

To formalize our encoding, we decided to use JSON Schema~\cite{handrewsjsonschema:18}. 
This defines a vocabulary allowing us to validate and annotate JSON documents. 
Additionally, tools for programatic verification exist in many languages. 

Unfortunatly, JSON schema is often tedious to write and read for humans. 
This is especially true when it comes to recursivly defined data strucutures.
OpenMath has many recusrive structures.
Instead of writing our encoding in JSON Schema directly, we decided to write the schema in a different language and then compile it to JSON Schema. 

For this purpose, we decided to make use of TypeScript~\cite{typescript:webpage}. 
TypeScript is a language derived from JavaScript -- TypeScript files are JavaScript plus type annotations. 
As such, it can be easily written and understood by humans. 
On top of typescript, we make use of a compiler~\cite{vega-ts-jscon-schema-generator:webpage} from TypeScript definitions into JSON Schema. 

\subsection{Encoding Details}

In general, objects in our encoding look similar to the following:
\begin{lstlisting}
{
    "kind": "OMV",
    "id": "something",
    "name": "x"
}
\end{lstlisting}

The \texttt{kind} attribute specifies which kind of OpenMath object this is. 
These values correspond to the element names used in the XML encoding. 
This correspondence lays the foundations of easy translation between the two. 
In TypeScript this property is also refered to as a \textit{Type Guard}, because if \textit{guards} the type of object that is represented. 

Like in the XML encoding it is possible to make use of structure sharing. 
For this purpose the \texttt{id} attribute can be used. 
We will come back to this in more detail below, when we define to the \texttt{OMR} type.

In the following we will go over the details of our encoding. 
For this we will make use of a TypeScript-like syntax, that is easily readable. 
In our description we omit the \texttt{id} attribute, which can be added to any encoded object. 
The complete source code of our encoding -- and details on how to use it -- can be found on Github~\cite{URL:openmathjson:github}. 

\subsubsection{Object Constructor -- OMOBJ}

The OpenMath Object Constructor -- \texttt{OMOBJ} -- is defined as follows:
\begin{lstlisting}
{
    "kind": "OMOBJ",
    /** optional version of openmath being used */
    "openmath": "2.0",
    /** the actual object */
    "object": omel /* any element */
}
\end{lstlisting}
Concretly, the integer 3 encapsulated in an object constructor using this encoding is as follows:
\begin{lstlisting}
{
    "kind": "OMOBJ",
    "openmath": "2.0",
    "object": {
        "kind": "OMI", 
        "integer": 3
    }
}
\end{lstlisting}

Let us have a look at this first example attribute for attribute. 

The first attribute -- \texttt{kind} -- represents the type of OpenMath object in question. 
Notice that it occurs twice -- once in the \texttt{OMOBJ} and a second time in the wrapped \texttt{OMI}. 
We will talk in detail about integer representation below, and hence only care about this first one. 

The second attribute -- \texttt{openmath} -- is defined as optional by our schema. 
This indicates the version of OpenMath that is being used -- \lstinline{"2.0"} in our case. 

The third and final attribute is the \texttt{object} attribute. 
This contains the wrapped object -- it is defined as of \texttt{omel} type. \ednote{Make sure the document is in this order?}
This type \texttt{omel} can contain any OpenMath element -- concretly
primitive objects (Symbols \texttt{OMS}, Variables \texttt{OMV}, Integers \texttt{OMI}, Floats \texttt{OMF}, Bytes \texttt{OMB}, Strings \texttt{OMSTR}), 
complex elements (Application \texttt{OMA}, Attribution \texttt{OMATTR}, Binding \texttt{OMBIND}) or
Errors \texttt{OME} and References \texttt{OMR}. 
In this particular case, we just have the integer $3$. 

\subsubsection{Symbols -- OMS}

An OpenMath Symbol is encoded as follows:
\begin{lstlisting}
{
    "kind": "OMS",
    /** the base for the cd, optional */
    "cdbase": uri, /* any valid URI */, 
    /** content dictonary the symbol is in, any uri */
    "cd": uri,
    /** name of the symbol */
    "name": name /* any valid symbol name */
}
\end{lstlisting}

Notice the \texttt{uri} and \texttt{name} types in the definition. 
These are not directly JSON types. 
We define the \texttt{uri} type to be a any JSON string that represents a valid URI. 
Similarly, we define the \texttt{name} type to be any JSON string that represents a valid symbol name. 

For example to encode the \texttt{sin} symbol from the \texttt{transc1} CD:
\begin{lstlisting}
{
    "kind": "OMS",
    "cd": "transc1",
    "name": "sin"
}
\end{lstlisting}

\subsubsection{Variables -- OMV}

An OpenMath Variable is encoded as follows:
\begin{lstlisting}
{
    "kind": "OMV",
    /** name of the variable */
    "name": name
}
\end{lstlisting}

We again make use of the \texttt{name} type here. 

For example to encode a variable $x$:
\begin{lstlisting}
{
    "kind": "OMV",
    "name": "x"
}
\end{lstlisting}

\subsubsection{Integers -- OMI}

Unlike the previous elements, our encoding allows integers to be encoded in three different ways. 
In particular, we define them as follows:
\begin{lstlisting}
{
    "kind": "OMI",
    //
    // exactly one of the following
    //

    /* any json integer */
    "integer": integer,
    /* any string matching ^-?[0-9]+$ */
    "decimal": decimalInteger,
    /* any string matching ^-?x[0-9A-F]+.$ */
    "hexadecimal": hexInteger
}
\end{lstlisting}

We allows integers to be represented as one of the following:
\begin{description}
    \item[JSON Integers]
    Representing an OpenMath Integer as a JSON integer allows making use of datatypes that JSON offers. 
    This was one of the goals we wanted to achieve with our encoding. 
    
    JSON has no integer type in and of itself -- it only provides a \texttt{number} type. 
    To work around this in our schema, we define a custom integer type as any number that is an integer. 
        
    OpenMath integers can be of arbitrary size. 
    While the JSON specification does not limit the size of numbers, it also allows any implementation the freedom to pick some limit. 
    Thus for reasonably sized integers, this representation works well. 
    But for larger numbers, the other two variants should be used. 

    \item[Decimal Strings]
    A second way to encode an OpenMath Integer is to make use of the straightforward decimal string encoding. 
    This is any string consisting only of digits (and potentially having a minus sign in the case of a negative integer). 

    \item[Hexademical String]
    We can allow to encode an OpenMath Integer as a hexadecimal integer. 
    To make this different from decimmally encoded integers, we require a lower-case x to be placed before the string. 
    This also has the advantage that this corresponds exactly to the XML encoding.     
\end{description}

We can thus encode the integer $-120$ in three different ways:
\begin{itemize}
    \item as a JSON integer
\begin{lstlisting}
{
    "kind": "OMI",
    "integer": -120
}
\end{lstlisting}
    \item as a decimal-encoded string
\begin{lstlisting}
{
    "kind": "OMI",
    "decimal": "-120"
}
\end{lstlisting}
    \item as a hexadecimal-encoded string
\begin{lstlisting}
{
    "kind": "OMI",
    "hexadecimal": "-x78"
}
\end{lstlisting}
\end{itemize}

\subsubsection{Floats -- OMF}

Like integers, we allow floats to be encoded in three different ways:
\begin{lstlisting}
{
    "kind": "OMF",

    //
    // exactly one of the following
    //

    /* any json number */
    "float": number,
    /* any string matching 
        (-?)([0-9]+)?(\.[0-9]+)?([eE](-?)[0-9]+)? */
    "decimal": decimalFloat,
    /* any string matching ^([0-9A-F]+)$ */
    "hexadecimal": hexFloat 
}
\end{lstlisting}

Here, the different cases work exactly like in the integer case. 

We can represent a float either as a native JSON number, a string in decimal representation or a string in hexadecimal representation. 
Just like above, the string representations correspond to the ones allowed by the XML encoding. 

For example the floating point number $10^{-10}$ can be represented in three different ways:
\begin{itemize}
    \item as a JSON float
\begin{lstlisting}
{
    "kind": "OMF",
    "float": 1e-10
}
\end{lstlisting}
    \item as a decimal-encoded string
\begin{lstlisting}
{
    "kind": "OMF",
    "decimal": "0.0000000001"
}
\end{lstlisting}
    \item as a hexadecimal-encoded string
\begin{lstlisting}
{
    "kind": "OMF",
    "hexaecimal": "3DDB7CDFD9D7BDBB"
}
\end{lstlisting}
\end{itemize}

\subsubsection{Bytes -- OMB}

Bytes can be encoded in two different ways:
\begin{lstlisting}
{
    "kind": "OMB",
    //
    // exactly one of the following
    //

    /** an array of bytes
        where a byte is an integer from 0 to 255 */
    "bytes": byte[],
    /** a base64 encoded string */
    "base64": base64string
}
\end{lstlisting}

\begin{description}
    \item[Array of bytes]
    This encoding again makes use of JSON datastructures -- representing bytes as a concrete list of bytes.
    As the byte datatype does not exist directly in JSON, we represent a single byte as an integer between 0 and 255 (inclusive). 

    \item[Base64 encoded string]
    A base64-encoded string of bytes corresponds to the XML encoding. 
\end{description}

For example we can encode the ascii bytes of the string \textit{hello world}
\begin{itemize}
    \item as a byte array
\begin{lstlisting}
{
    "kind": "OMB",
    "bytes": [
        104, 101, 108, 108, 111, 32, 
        119, 111, 114, 108, 100
    ]
}
\end{lstlisting}
    \item as a base64-encoded string
\begin{lstlisting}
{
    "kind": "OMB",
    "base64": "aGVsbG8gd29ybGQ="
}
\end{lstlisting}
\end{itemize}

\subsubsection{Strings -- OMS}
The encoding of strings is straightforward -- we can make use of native JSON strings. 
\begin{lstlisting}
{
    "kind": "OMSTR", 
    /** the string */
    "string": string
}
\end{lstlisting}

Thus the string \lstinline{"Hello world"} is encoded as follows:
\begin{lstlisting}
{
    "kind": "OMSTR", 
    "string": "Hello world"
}
\end{lstlisting}




\subsection{The Demo Site}

To demonstrate our OpenMath-JSON encoding, we have created a demo site which can be found at~\cite{openmathjson:web}. 
This site is implemented in TypeScript and encapsulated using Docker\cite{docker:webpage}. 

The demo site serves three purposes.

Primarily, it serves as a presentation of the encoding, providing examples and documenting it's usage. 

Secondly, it enables validation of OpenMath JSON objects. This can be seen in Figure\ednote{Make figure}.
The user can enter some JSON, press the \textit{Validate JSON} button, and receive immediate feedback if their JSON is a valid OpenMath object or not. 
In particular, the user can also see a detailed error message if their object is not valid OpenMath JSON. 

This makes use of the OpenMath JSON schema, and validates the users' JSON using a generic JSON Schema Validator. 
Furthermore, this is also exposed using a REST API, enabling easy validation of OpenMath JSON in other applications. 

Thirdly, it enables translation between XML and JSON OpenMath objects. 
Like for validation, the site enables the user to enter some JSON and be presented with some XML and vice-versa. 
This can be seen in Figure\ednote{Make figure + screenshot}.

As we designed our encoding with this translatability goal in mind, the implementation of it was straight-forward. 
Nonetheless, this translation is also exposed using a REST API. 