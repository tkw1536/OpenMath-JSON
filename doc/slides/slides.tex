\documentclass[usenames,dvipsnames]{beamer}

\usetheme{Madrid}

\makeatletter
\setbeamertemplate{footline}{%
  \leavevmode%
  \hbox{%
  \begin{beamercolorbox}[wd=.2\paperwidth,ht=2.25ex,dp=1ex,center]{author in head/foot}%
    \usebeamerfont{author in head/foot}\insertshortauthor\expandafter\beamer@ifempty\expandafter{\beamer@shortinstitute}{}{~~(\insertshortinstitute)}
  \end{beamercolorbox}%
  \begin{beamercolorbox}[wd=.4\paperwidth,ht=2.25ex,dp=1ex,center]{title in head/foot}%
    \usebeamerfont{title in head/foot}\insertshorttitle
  \end{beamercolorbox}%
  \begin{beamercolorbox}[wd=.4\paperwidth,ht=2.25ex,dp=1ex,right]{date in head/foot}%
    \usebeamerfont{date in head/foot}\insertshortdate{}\hspace*{2em}
    \insertframenumber{} / \inserttotalframenumber\hspace*{2ex} 
  \end{beamercolorbox}}%
  \vskip0pt%
}
\makeatother

\usepackage{mathtools} % for over- and underbrackets

% listings syntax highlighting for JSOn
\usepackage{listings}

\colorlet{bool}{red}
\colorlet{string}{blue}
\colorlet{number}{orange}
\colorlet{comment}{OliveGreen}
\colorlet{type}{purple}

\lstset{
    basicstyle=\small\ttfamily,
    columns=fullflexible,
    % strings
    string=[s]{"}{"},
    stringstyle=\color{string},
    % booleans
    keywords={true,false},
    keywordstyle=\color{bool},
    % comments
    comment=[s]{/*}{*/},
    morecomment=[l]{//},
    commentstyle=\color{comment},
    % types
    morekeywords=[2]{base64string,omel,uri,name,any,string,integer,decimalInteger,hexInteger,float,decimalFloat,hexFloat,byte,OMFOREIGN,OMV,attvar},
    keywordstyle=[2]{\color{type}},
}

% digits
\lstset{literate=%
   *{0}{{{\color{number}0}}}1
    {1}{{{\color{number}1}}}1
    {2}{{{\color{number}2}}}1
    {3}{{{\color{number}3}}}1
    {4}{{{\color{number}4}}}1
    {5}{{{\color{number}5}}}1
    {6}{{{\color{number}6}}}1
    {7}{{{\color{number}7}}}1
    {8}{{{\color{number}8}}}1
    {9}{{{\color{number}9}}}1
    {e-}{{{\color{number}e-}}}2
    {-}{{{\color{number}-}}}1
    {.}{{{\color{number}.}}}1
    {base64string}{{{\color{type}base64string}}}{12}%keyword gets caught by number, so escape it
}

% lstinline will be of normal size
\makeatletter
\let\lstinline@org\lstinline
\def\lstinline{\lstinline@org[basicstyle=\ttfamily]}
\makeatother

\begin{document}

% TODO: Add CICM workshop details etc
\title{A Proposal for an OpenMath JSON Encoding}
\author[Tom Wiesing et al]{\underline{Tom Wiesing}\and Michael Kohlhase\\Computer Science, FAU Erlangen-N{\"u}rnberg}
\date[July 10 2019, CICM Prague]{July 10, 2019\\OpenMath Workshop\\Conference on Intelligent Computer Mathematics\\Prague, Czech Republic}


\begin{frame}
    \titlepage
\end{frame}

% =================================================================
% Motivation
% =================================================================
\begin{frame}[fragile]
    \frametitle{What is JSON?}
    \begin{itemize}
        \item JSON = \textbf{J}ava\textbf{S}cript \textbf{O}bject \textbf{N}otation%json.org
        \begin{itemize}
            \item lightweight data-interchange format
            \item subset of JavaScript (used a lot on the web)
            \item defined independently
        \end{itemize}
        \item Primitive types
        \begin{itemize}
                \item Strings (e.g. \lstinline{"Hello world"})
                \item Numbers (e.g. \lstinline{42} or \lstinline{3.14159265})
                \item Booleans (\lstinline{true} and \lstinline{false})
                \item \lstinline{null}
        \end{itemize}
        \item Composite types
        \begin{itemize}
            \item Arrays (e.g. \lstinline{[1, "two", false]})
            \item Objects (e.g. \lstinline|{"foo": "bar", "answer": 42}|)
        \end{itemize}
    \end{itemize}
\end{frame}

\begin{frame}
    \frametitle{Why an OpenMath encoding for JSON?}
    \begin{itemize}
        \item an OpenMath JSON encoding would make it easy to use across many languages
        \begin{itemize}
            \item JSON support exists in most modern programming languages
            \begin{itemize}
                \item corresponding native types common
                \item serialization to/from JSON without external library
            \end{itemize}
        \end{itemize}
        \item some existing approaches for an OpenMath JSON encoding
        \begin{itemize}
            \item discussed / suggested on the OpenMath mailing list
            \item we will look at two examples here
        \end{itemize}
    \end{itemize}
\end{frame}

\begin{frame}[fragile]
    \frametitle{XML as JSON}
    \begin{itemize}
        \item \textbf{Idea:} Generically encode XML as JSON
        \item use the JSONML standard for this%http://www.jsonml.org/
        \item e.g. $\mathrm{plus}(x, 5)$ corresponds to:
\begin{lstlisting}
[
    "OMOBJ",
    {"xmlns":"http://www.openmath.org/OpenMath"},
    [
        "OMA",
        ["OMS", {"cd": "arith1", "name": "plus"}],
        ["OMV", {"name": "x"}],
        ["OMI", "5"]
    ]
]
\end{lstlisting}
    \end{itemize}
\end{frame}

\begin{frame}[fragile]
    \frametitle{XML as JSON (2)}
    \begin{itemize}
        \item Advantages
        \begin{itemize}
            \item based on well-known XML encoding
            \item easy to understand based on it
        \end{itemize} 
        \item does not make use of JSON structures
        \begin{itemize}
            \item all attributes are encoded as strings, even numbers
            \item e.g. \lstinline{1e-10} (a valid JSON literal) can not be used
        \end{itemize}
        \item retains some of the XML akwardness
        \begin{itemize}
            \item introduces unnecessary overhead
            \item e.g. some pseudo-elements (such as OMATP) are needed
        \end{itemize}
    \end{itemize}
\end{frame}


\begin{frame}[fragile]
    \frametitle{OpenMath-JS}%https://github.com/lurchmath/openmath-js
    \begin{itemize}
        \item OpenMath-JS
        \begin{itemize}
            \item an (incomplete) implementation of OpenMath in JavaScript
            \item developed by Nathan Carter for use with Lurch Math on the web
            \item written in literate coffee script, a derivative language of JavaScript
        \end{itemize}
        \item e.g. $\mathrm{plus}(x, 5)$ corresponds to:
\begin{lstlisting}
{
    "t": "a",
    "c": [  
      {"t": "sy", "cd": "arith1", "n": "plus"}, 
      {"t": "v", "n": "x"}, 
      {"t": "i", "v": "5"}
    ]
}
\end{lstlisting}
    \end{itemize}
\end{frame}

\begin{frame}[fragile]
    \frametitle{OpenMath-JS (2)}
    \begin{itemize}
        \item does make use of JSON native structures
        \begin{itemize}
            \item much better than \textit{JSON-ML}
            \item small property names keep size of transmitted objects small
        \end{itemize}
        \item comes with some problems
        \begin{itemize}
            \item hard to read for humans
            \item written for \textit{JavaScript}, not JSON
            \item no formal schema
        \end{itemize}
    \end{itemize}
\end{frame}


% =================================================================
% Our Approach for an OpenMath encoding
% =================================================================

\begin{frame}
    \frametitle{Towards an OpenMath JSON Formalization}
    \begin{itemize}
        \item we need to write a new OpenMath JSON encoding
        \begin{itemize}
            \item combine advantages of the above two
            \item should be close to the XML encoding
            \item should make use of JSON concepts
        \end{itemize}
        \item we want to formalize this JSON encoding
        \begin{itemize}
            \item to verify JSON objects
            \item not done by existing approaches
        \end{itemize}
        \item comes with some positive side effects
        \begin{itemize}
            \item formalization of JSON $\Rightarrow$ structure definition in most languages
            \item trivial to use advanced serialization tools
            \begin{itemize}
                \item e.g. \textit{Protocol Buffers}, \textit{ZeroMQ}
            \end{itemize}
        \end{itemize}
        \item we can use JSON Schema%http://json-schema.org/
        \begin{itemize}
            \item a vocabulary allowing us to validate and annotate JSON documents
            \item tools for verification exist
        \end{itemize}
    \end{itemize}
\end{frame}

\begin{frame}
    \frametitle{Towards an OpenMath JSON Formalization (2)}
    \begin{itemize}
        \item JSON schema is often tedious to write and read
        \begin{itemize}
            \item especially when it comes to recusrive data types
            \item but implementation of it still exist 
        \end{itemize}
        \item \textbf{Idea:} Write schema in a \textbf{TypeScript Definition file}, compile into a JSON schema
        \begin{itemize}
            \item TypeScript = JavaScript + Type Annotations
            \item easily writeable and understandable
            \item a compiler from TypeScript Definitions into JSON Schema exists
        \end{itemize}
        \item We have done this, will present some examples
    \end{itemize}
\end{frame}

\begin{frame}
    \frametitle{Towards an OpenMath JSON Formalization (3)}
    \begin{itemize}
        \item Wrote a JSON Schema
        \begin{itemize}
            \item was written as described above
            \item we will give an overview how this looks below
        \end{itemize}
        \item Wrote a translator from OpenMath XML to JSON (we have actually built two)
        \begin{enumerate}
            \item web demo on (\url{https://omjson.openmath.org})
            \item as part of MMT (i.e. Scala) in the form of a RESTful API
        \end{enumerate}
        \item Wrote an extension of the \textit{OpenMath Standard} to make it an official encoding
    \end{itemize}
\end{frame}

\begin{frame}[fragile]
    $\Rightarrow$ Let's look at the concrete proposal for examples
\end{frame}



\end{document}
